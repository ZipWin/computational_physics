\documentclass{article}
\usepackage{ctex}
\usepackage{bm}
\usepackage[colorlinks, linkcolor=blue]{hyperref}
\usepackage{geometry}
\geometry{left=3cm, right=3cm, top=3cm, bottom=3cm}
\usepackage{amsmath}
\usepackage{amsthm}
\usepackage[T1]{fontenc}
\usepackage{xcolor}
\usepackage{lmodern}
\usepackage{listings}

\newtheorem{task}{问题}

\lstset{
%	numbers=left, 
	numberstyle= \small, 
	keywordstyle= \color{ blue!70},
	commentstyle= \color{red!50!green!50!blue!50}, 
	frame=shadowbox, % 阴影效果
	showstringspaces = false,
	flexiblecolumns,                % 别问为什么,加上这个
	rulesepcolor= \color{ red!20!green!20!blue!20} ,
	escapeinside=``, % 英文分号中可写入中文
	xleftmargin=2em,xrightmargin=2em, aboveskip=1em,
	framexleftmargin=2em
}

\lstdefinestyle{Fortran}{
	language        =   [90]Fortran,
	basicstyle=\small\ttfamily
}

\title{作业二:非线性方程求根}
\author{英才1701 赵鹏威 U201710152}

\begin{document}
	\maketitle
	\tableofcontents
	\newpage
	\section{引言}
	在物理中会遇到很多方程求根的问题,一般这些方程是非线性的,甚至是超越方程. 另外,方程求根问题实际上就是求函数的零点,这也对应着求函数极值点的问题. 因此开发有效的数值求根方法是很有必要的. 常用的算法有:二分法、Jacobi 迭代法、牛顿下山法. 为了让 Jacobi 迭代法收敛更快,又有人提出了事后加速法、Atiken 加速法. 这次作业就是通过 Fortran 来实现这些算法. 
	
	\section{问题描述}
	\begin{task}
		使用不同的算法求非线性方程
		\[
		f(x)=\frac{x^3}{3}-x=0
		\]
		的根,并比较它们的性能,包括:结果的准确性、误差大小、迭代次数
	\end{task}
	
	这是一个一元三次方程,很容易得到这个方程的解析解
	\[
	x_1 = -\sqrt{3}\approx-1.73205 \quad x_2 = 0 \quad x_3=\sqrt{3}\approx1.73205.
	\]
	这里将解析解转换成了精确到5位小数的数值,之后会将由算法得到的结果与这个结果来比较,验证算法的准确性.
	
	\section{程序实现}
	希望程序达到的效果是:向程序提供函数$f(x)$(和迭代式$\varphi(x)$,如果有的话),区间$[a, b]$,程序可以调用某种指定的算法,在指定的精度下,找出这个区间内所有可能的根. 
	
	为了达到这个目的,通过两步来实现整个程序. 第一步是将每种算法分别单独写成一个 subroutine,这些 subroutine 以函数 $f$,迭代式 $\varphi$,初值(对二分法来说是初始的区间)和要求的精度为输入参数,返回最多一个找到的根. 第二步是另写一个 subroutine 作为调用这些算法的接口程序,它的作用是将输入的区间等分成若干个小区间,然后调用第一步中的子程序在这些小区间内寻找根,并且对于方便预先判断收敛性的算法,在调用之前会自动判断这个区间内的迭代是否一定会收敛,跳过不一定收敛的区间,这样可以省去很多不必要的计算.
	
	下面详细阐述实现细节.
	
	\subsection{辅助模块}
	这个模块存放一些对程序实现有帮助但不是必须的东西,包括一些常数、新的类型定义和一些函数.
	\subsection{result 类型}
	由于几乎在所有的情况下,数值计算的结果具有一定的误差,所以可以定义一个 result 类型,将计算结果和误差封装在一起储存.
	
	
\end{document}